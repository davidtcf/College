\documentclass[oneside]{book}
\input{~/Latex-Envi/theorems.tex}
\begin{document}
\title{PHYS142 Review Note}
\author{Chenfei Tang}
\maketitle

\chapter{Introduction, Coulomb Law, Energy}
\section{Basics of Electromagnetism}
Single unified force, our perception of a force as electric or magnetic depends on our \textbf{state of motion}.

\begin{center}
\textbf{Electric Fields created by:}
\begin{itemize}
    \centering
    \item Static Electric Charges
    \item TIme-Varying Magnetic Fields
\end{itemize}


\textbf{Magnetic Fields created by}
\begin{itemize}
    \centering
    \item Moving Electric Charges (Currents)
    \item Time-Varying Electric Fields
\end{itemize}

\textbf{Maxwell's Equation:}\\
    \centering
    $\nabla\cdot \Vec{E}=\frac{\rho}{\epsilon_0}$\\
    $\nabla\times \Vec{E}=-\frac{\partial \vec{B}}{\partial t}$\\
    $\nabla\cdot \Vec{B}=o$\\
    $\nabla\times \Vec{B}= \mu_0 \vec{J}+\mu_0 \epsilon_0 \frac{\partial \vec{E}}{\partial t}$
\end{center}

\section{Electrostatics:Charges At Rest}
\prop{Electric Charges}{
\begin{enumerate}
    \item Charge Conservation\\
      \begin{itemize}
          \item Total Q in isolated system never changes
            \item Isolated $\equiv$ particles cannot escape boundary
                      
      \end{itemize}
    \item Charge quantization
      \begin{itemize}
          \item All charges come in integer multipier of e
            \item No free quarks observed
            
      \end{itemize}   
\end{enumerate}
\section{Force on Charges:Coulomb Law}


}






\end{document}
